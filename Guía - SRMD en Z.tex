\documentclass[spanish,letterpaper, 11pt, addpoints, answers]{exam}
\usepackage[left=2cm, right=2cm, top=2cm, bottom=2.5cm]{geometry}
\PassOptionsToPackage{T1}{fontenc} 
    \usepackage{fontenc} 
    \usepackage[utf8]{inputenc}
    % Cargar babel y configurar para español
    \usepackage[spanish, es-tabla, es-noshorthands]{babel}
    \usepackage{lmodern}
    \usepackage{amsfonts}
    \usepackage{multirow}
    \usepackage{hhline}
    \usepackage[none]{hyphenat}
\usepackage[utf8]{inputenc}
\usepackage{graphics}
\usepackage{color}
\usepackage{amssymb}
\usepackage{amsmath}
\usepackage{enumitem}
\usepackage{xcolor}
\usepackage{cancel}
\usepackage{ragged2e}
\usepackage{graphicx}
\usepackage{multicol}
\usepackage{color}
\usepackage{tikz}
\pointpoints{Punto}{Puntos}

\usepackage{graphicx}
\usepackage{tikz}
\usetikzlibrary{babel,arrows.meta,decorations.pathmorphing, backgrounds,positioning,fit,petri, shapes, shadows}

\usepackage{tikz,color}
\usepackage{pgf-pie} 

\usepackage{stackengine}
\newcommand\xrowht[2][0]{\addstackgap[.5\dimexpr#2\relax]{\vphantom{#1}}}



\CorrectChoiceEmphasis{\color{red}}
\noprintanswers

\renewcommand{\choiceshook}{%
    \setlength{\leftmargin}{30pt}%
}


\everymath={\displaystyle}
\renewcommand{\choicelabel}{\thechoice)}
\renewcommand{\choicelabel}{ 
  \ifnum\value{choice}>0
    \makebox[1.5cm][r]{\raggedleft \thechoice)}
  \else
   \raggedleft \thechoice)
  \fi
}

\setlength{\multicolsep}{0.6em}

\setlength{\columnseprule}{2pt}

\setlength{\columnsep}{2cm}
%%%%%% ---Comment out to add a header image ----

\begin{document}
%\begin{figure}[t]
%\includegraphics[width=1\textwidth,height=1.2\textheight,keepaspectratio]{header-cufm.png}
%\end{figure}

\begin{center}
\textbf{Guía de ejercitación} \\
Operatoria combinada de enteros\\
2do semestre 2023
\end{center}
\extraheadheight{-0.5in}

\runningheadrule \extraheadheight{0.15in}

\vspace{0.15in}
\runningheadrule \extraheadheight{0.14in}

\lhead{\ifcontinuation{Pregunta \ContinuedQuestion\ continua\ldots}{}}
\runningheader{Operatoria combinada en $\mathbb{Z}$}{Guía de ejercitación}{2do semestre 2023}
\runningfooter{}
              {\thepage\ de \numpages}
              {}
\vspace{0.05in}

\nopointsinmargin
\setlength\linefillthickness{0.1pt}
\setlength\answerlinelength{0.1in}
\vspace{0.1in}
%\parbox{6in}{
%\textbf{Objetivo:} Verificar el aprendizaje y comprensión de algunos de los temas fundamentales del nivel. }
%\vspace{0.15in}
\hrule 

\begin{questions}

\question Calcula las siguientes multiplicaciónes y divisiones de números enteros

\begin{itemize}
\item[a. ]$8\cdot (-9) =$
\item[b. ]$-12\cdot 5\div (-6) =$
\item[c. ]$(-1)\cdot 3\cdot (-6)\cdot (-5)\cdot (-4)=$
\item[d. ]$(-15)\cdot 24\div(20)\cdot (-9)\div (-6) =$
\item[e. ]$\underbrace{(-1)\cdot (-1)\cdot (-1)\cdot \ldots \cdot (-1)}_{\text{12638 veces}} =$
\end{itemize}

\question Resuelve la siguiente operatoria combinada de números enteros.

\textbf{Nivel 1.}
\begin{itemize}
\item[a.] $6\cdot(-3)+5=$.

\item[b.] $-27\div (-3)\cdot 3+(-3)=$

\item[c.] $-13+4\cdot (-3)-15\div (5)=$

\item[d.] $-1-1-1\cdot(-1)+1=$

\item[e.] $2+(-2)\div (2)+5\cdot (-2)=$
\end{itemize}

\textbf{Nivel 2.}
\begin{itemize}
\item[a.] $-5+(-3)\cdot 3-(-12)\div (-2)=$.

\item[b.] $7\cdot 4\cdot (-6)\div (-12)+(-6)\div 2\cdot (-3)=$

\item[c.] $-1\cdot (-1)+1\cdot (-1)\div (-1)-(-1)\cdot (-1)-1=$

\item[d.] $(-2)\cdot (-5)\div (-2)-(-18)\div(-3)\cdot (-2)+(-12)=$

\item[e.] $(-6)\cdot(-5)+6\div(-6)-(-4)=$

\end{itemize}

\textbf{Nivel 3.}
\begin{itemize}
\item[a.] $-7+(-3\cdot (-5)+1)\div(-4)=$.

\item[b.] $-(-6\cdot (-7)+6\cdot(-3))\div 8-4=$

\item[c.] $(-8\cdot (-12))\div (-5+(-3))+(-4\cdot 3)=$

\item[d.] $2+(9-6-15-3)\div (5+(-7)-13+9-(-1))=$ 

\item[e.] $-(-15\cdot (-18)-2\cdot 7)\div (-2)\div (-8)-16=$ 

\end{itemize}

\textbf{Nivel 4.}
\begin{itemize}
\item[a.] $(6-(-4\cdot 6)\div 8)\div 3+(-7)\cdot 3=$

\item[b.] $(5\cdot (-11+(-7)))\cdot(-3+5\cdot 2)\div((-3\cdot 5+6)\div (-2-1)+11)=$

\item[c.] $((-2)\cdot (-5+(-8)))\div (-2)-(-18)\div(-3\cdot (-2)+(-12))=$

\item[d.] $((-5+(-7)\cdot 8)+5)\div (-7)-(-3\cdot (-12 )\div (-9))=$

\item[e.] $-(-(8-6\cdot (-4))+(-3))\div ((-5\cdot 3-7)\div(-2)+(-4))=$

\end{itemize}
\end{questions}
\end{document}