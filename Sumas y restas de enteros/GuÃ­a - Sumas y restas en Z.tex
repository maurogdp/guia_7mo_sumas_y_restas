\documentclass[spanish,letterpaper, 11pt, addpoints, answers]{exam}
\usepackage[left=2cm, right=2cm, top=2cm, bottom=2.5cm]{geometry}
\PassOptionsToPackage{T1}{fontenc} 
    \usepackage{fontenc} 
    \usepackage[utf8]{inputenc}
    % Cargar babel y configurar para español
    \usepackage[spanish, es-tabla, es-noshorthands]{babel}
    \usepackage{lmodern}
    \usepackage{amsfonts}
    \usepackage{multirow}
    \usepackage{hhline}
    \usepackage[none]{hyphenat}
\usepackage[utf8]{inputenc}
\usepackage{graphics}
\usepackage{color}
\usepackage{amssymb}
\usepackage{amsmath}
\usepackage{enumitem}
\usepackage{xcolor}
\usepackage{cancel}
\usepackage{ragged2e}
\usepackage{graphicx}
\usepackage{multicol}
\usepackage{color}
\usepackage{tikz}
\pointpoints{Punto}{Puntos}

\usepackage{graphicx}
\usepackage{tikz}
\usetikzlibrary{babel,arrows.meta,decorations.pathmorphing, backgrounds,positioning,fit,petri, shapes, shadows}

\usepackage{tikz,color}
\usepackage{pgf-pie} 

\usepackage{stackengine}
\newcommand\xrowht[2][0]{\addstackgap[.5\dimexpr#2\relax]{\vphantom{#1}}}



\CorrectChoiceEmphasis{\color{red}}
\noprintanswers

\renewcommand{\choiceshook}{%
    \setlength{\leftmargin}{30pt}%
}


\everymath={\displaystyle}
\renewcommand{\choicelabel}{\thechoice)}
\renewcommand{\choicelabel}{ 
  \ifnum\value{choice}>0
    \makebox[1.5cm][r]{\raggedleft \thechoice)}
  \else
   \raggedleft \thechoice)
  \fi
}

\setlength{\multicolsep}{0.6em}

\setlength{\columnseprule}{2pt}

\setlength{\columnsep}{2cm}
%%%%%% ---Comment out to add a header image ----

\begin{document}
%\begin{figure}[t]
%\includegraphics[width=1\textwidth,height=1.2\textheight,keepaspectratio]{header-cufm.png}
%\end{figure}

\begin{center}
\textbf{Guía de ejercitación} \\
Sumas y restas de enteros\\
2do semestre 2023
\end{center}
\extraheadheight{-0.5in}

\runningheadrule \extraheadheight{0.15in}

\vspace{0.15in}
\runningheadrule \extraheadheight{0.14in}

\lhead{\ifcontinuation{Pregunta \ContinuedQuestion\ continua\ldots}{}}
\runningheader{Sumas y restas en $\mathbb{Z}$}{Guía de ejercitación}{2do semestre 2023}
\runningfooter{}
              {\thepage\ de \numpages}
              {}
\vspace{0.05in}

\nopointsinmargin
\setlength\linefillthickness{0.1pt}
\setlength\answerlinelength{0.1in}
\vspace{0.1in}
%\parbox{6in}{
%\textbf{Objetivo:} Verificar el aprendizaje y comprensión de algunos de los temas fundamentales del nivel. }
%\vspace{0.15in}
\hrule 

\begin{questions}

\question Calcula las siguientes sumas de números enteros
\begin{multicols}{2}
\begin{itemize}
\item[a. ]$-5+(-7) =$
\item[b. ]$4+(-11) =$
\item[c. ]$-7+(-2)+(8) =$
\item[d. ]$-3+7+(5)+(-9) =$
\item[e. ]$-13+(-15)+16+(-4)=$
\end{itemize}
\end{multicols}

\question Calcula las siguientes restas de números enteros.
\begin{multicols}{2}
\begin{itemize}
\item[a.] $3-8=$.

\item[b.] $-9-5=$

\item[c.] $-4-13-(-5)=$

\item[d.] $-7-14-(-6)-(-2)-18=$

\item[e.] $18-(-7)-13-19-(-7)-(5)=$


\end{itemize}
\end{multicols}

\question Resuelve las siguientes sumas y restas de enteros.
\begin{multicols}{2}
\textbf{Nivel 1.}\\
\begin{itemize}
\item[a.] $5-3+9=$.

\item[b.] $-5+6-3+8-7=$

\item[c.] $7-3-9+5+2-9=$
\end{itemize}

\textbf{Nivel 2.}\\
\begin{itemize}
\item[d.] $-5+(-7)-(-3)+7-(7)=$

\item[e.] $-8-9-(-4)+3-(-5)+(-7)=$

\item[f.] $(-13)+(-17)+6-(-2)+1-(+6)=$
\end{itemize}
\end{multicols}

\textbf{Nivel 3.}\\
\begin{itemize}
\item[g.] $-3+(-5+4-(-3))=$

\item[h.] $7-(-5+6-9)+5-7+(4-17+(-8))=$

\item[i.] $-8+(-5-(-4)+(-6)+9)-(-6)+15+(-12+5-11-(-16))=$ 
\end{itemize}

\textbf{Nivel 4.}\\
\begin{itemize}
\item[g.] $-6+(-3+5-4+(-3+5))=$

\item[h.] $-(-3-(-5+(-7+4)))=$

\item[i.] $9-(+7-3+(-4-15+7)-6-(-2-6-8+5))-(-4)+(-(+(-(-8))))=$ 
\end{itemize}

\textbf{Nivel 5.}\\
\begin{itemize}
\item[g.] $|-5+3|-4+(-7)-|-7+13-2|-1=$

\item[h.] $-|-7+(-3+5-|-3+7|+5)|+(-3+9)=$

\item[i.] $|5-(-3+5-|-2+6-(-8-7-12+|-3+9-10|+5)|)-8+(-13-|6-7+1|)|=$ 
\end{itemize}

\end{questions}
\end{document}