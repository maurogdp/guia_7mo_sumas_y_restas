\documentclass[spanish,letterpaper, 12pt, addpoints, answers]{exam}
\usepackage[left=2cm, right=2cm, top=2cm, bottom=2.5cm]{geometry}
\PassOptionsToPackage{T1}{fontenc} 
    \usepackage{fontenc} 
    \usepackage[utf8]{inputenc}
    % Cargar babel y configurar para español
    \usepackage[spanish, es-tabla, es-noshorthands]{babel}
    \usepackage{lmodern}
    \usepackage{amsfonts}
    \usepackage{multirow}
    \usepackage{hhline}
    \usepackage[none]{hyphenat}
\usepackage[utf8]{inputenc}
\usepackage{graphics}
\usepackage{color}
\usepackage{amssymb}
\usepackage{amsmath}
\usepackage{enumitem}
\usepackage{xcolor}
\usepackage{cancel}
\usepackage{ragged2e}
\usepackage{graphicx}
\usepackage{multicol}
\usepackage{color}
\usepackage{tikz}
\pointpoints{Punto}{Puntos}

\usepackage{graphicx}
\usepackage{tikz}
\usetikzlibrary{babel,arrows.meta,decorations.pathmorphing, backgrounds,positioning,fit,petri, shapes, shadows}

\usepackage{tikz,color}
\usepackage{pgf-pie} 

\usepackage{stackengine}
\newcommand\xrowht[2][0]{\addstackgap[.5\dimexpr#2\relax]{\vphantom{#1}}}

\CorrectChoiceEmphasis{\color{red}}
\noprintanswers

\renewcommand{\choiceshook}{%
    \setlength{\leftmargin}{30pt}%
}


\everymath={\displaystyle}
\renewcommand{\choicelabel}{\thechoice)}
\renewcommand{\choicelabel}{ 
  \ifnum\value{choice}>0
    \makebox[1.5cm][r]{\raggedleft \thechoice)}
  \else
   \raggedleft \thechoice)
  \fi
}

\setlength{\multicolsep}{0.6em}

\setlength{\columnseprule}{2pt}

\setlength{\columnsep}{2cm}
%%%%%% ---Comment out to add a header image ----

\begin{document}
%\begin{figure}[t]
%\includegraphics[width=1\textwidth,height=1.2\textheight,keepaspectratio]{header-cufm.png}
%\end{figure}

\begin{center}
\textbf{Evaluación N$^o01$ Matemáticas} \\
Séptimo básico \\
2do semestre 2023
\end{center}
\extraheadheight{-0.5in}

\textbf{Nombre:}\rule{9cm}{0.5pt}\hspace{1cm}\textbf{Curso:}\rule{2cm}{0.5pt}

\runningheadrule \extraheadheight{0.15in}

\vspace{0.15in}
\runningheadrule \extraheadheight{0.14in}

\lhead{\ifcontinuation{Pregunta \ContinuedQuestion\ continua\ldots}{}}
\runningheader{Matemáticas}{Evaluación N$^o01$}{2do semestre 2023}
\runningfooter{}
              {\thepage\ de \numpages}
              {}
\vspace{0.15in}

\nopointsinmargin
\setlength\linefillthickness{0.1pt}
\setlength\answerlinelength{0.1in}
\vspace{0.1in}
%\parbox{6in}{
%\textbf{Objetivo:} Verificar el aprendizaje y comprensión de algunos de los temas fundamentales del nivel. }
\vspace{0.1in}

 \begin{center}
\fbox{\fbox{\parbox{14cm}{
{\textbf{Instrucciones:}
\begin{itemize}
    \item Tienes 40 minutos para contestar esta evaluación.
    
    \item Lee cada pregunta cuidadosamente. Pon atención a los detalles. 
    
    \item Si terminas antes del tiempo, entonces aprovecha de revisar tus respuestas.
    %\item Solo en la \textbf{Parte III} de esta evaluación el desarrollo es obligatorio para obtener la totalidad del puntaje. 

\end{itemize} }}}}
\vspace{0.2in}
\end{center}

\parbox{16cm}{
{\textsc{\textbf{Ejercicios de desarrollo.}}}}

\vspace{0.15in}
\hrule 
%\vspace{0.35in}
\begin{questions}
%\begin{multicols}{2}


\question[3]  $4-5+3=$\\
\vspace{1cm}

\question[3]  $-3+9-6+2-7=$\\
\vspace{1cm}

\question[3]  $-5+(-8)-(-7)-4=$\\
\vspace{2cm}

\question[3]  $(-6)-(2)+15-13-(-9)+(7)=$\\
\vspace{2cm}

\question[3]  $(8-15+3)-(-7)+4=$\\
\vspace{3cm}

\question[3]  $7-9-(-5+8+3-2)+(-4-(-8))=$\\
\vspace{3cm}

\question[3]  $-(-7-(4-11+5)-3+9)-(-2)=$\\
\vspace{3cm}

\question[3]  $1-(-1+(-1-9-(3+(-7)-2)))=$\\
\vspace{3cm}

\question[3]  $12-5-|-4+8|-7+|6-9|=$\\
\vspace{3cm}

\question[3]  $-3+(-7+5-|4-2-(-12)|-3)+4-|-15|+5=$\\
\vspace{3cm}

\end{questions}
\end{document}